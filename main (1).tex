\documentclass[12pt,a4paper]{article}
\UseRawInputEncoding
\usepackage{amsmath, amssymb, amsthm}
\usepackage{geometry}
\usepackage{setspace}
\usepackage{hyperref}
\usepackage{lipsum}
\usepackage{graphicx}
\usepackage{listings}
\usepackage{xcolor}
\usepackage{titlesec}

\geometry{margin=1in}
\doublespacing

\title{\textbf{Compound Poisson Process: Theory, Distribution, Sensitivity and R Shiny Implementation}}
\author{}
\date{}

% Code formatting settings
\lstset{
    basicstyle=\ttfamily\small,
    breaklines=true,
    keywordstyle=\color{blue},
    commentstyle=\color{green!50!black},
    stringstyle=\color{red},
    frame=single
}

\begin{document}

\maketitle

\section*{Abstract}
This document presents the theoretical derivation, distributional properties, and sensitivity analysis of a Compound Poisson Process with exponential interarrival times and exponential jump sizes. The complete R Shiny code for interactive simulation is also included.

\section{Introduction}
A Compound Poisson Process is a fundamental stochastic model widely used in insurance, finance, queueing systems, and reliability engineering. It describes the cumulative impact of random events occurring over time, where each event contributes a random size.

We assume:
\begin{itemize}
    \item Interarrival times follow an exponential distribution with rate $\lambda$.
    \item Jump sizes $X_i$ are i.i.d.\ exponential with parameter $\mu$.
\end{itemize}

\section{Theory of the Compound Poisson Process}

Let $N(t)$ be a Poisson process with rate $\lambda$. Define the compound process:
\[
S(t) = \sum_{i=1}^{N(t)} X_i
\]
where $X_i$ are independent exponential random variables with rate $\mu$, independent of $N(t)$.

\subsection*{Distribution of $S(t)$}

Since
\[
N(t) \sim \mathrm{Poisson}(\lambda t),
\]
we have:
\[
P(N(t) = n) = e^{-\lambda t} \frac{(\lambda t)^n}{n!}.
\]

Conditional on $N(t)=n$,
\[
S(t) \mid N(t)=n \sim \mathrm{Gamma}(n, \mu)
\]
because the sum of $n$ i.i.d.\ exponential($\mu$) random variables is Gamma distributed.

Thus, $S(t)$ is a Poisson-Gamma mixture:
\[
P(S(t) \le s)
= \sum_{n=0}^{\infty} e^{-\lambda t}\frac{(\lambda t)^n}{n!} 
\, F_{\Gamma(n,\mu)}(s).
\]

\subsection*{Mean and Variance}
\[
\mathbb{E}[S(t)] = \lambda t \cdot \frac{1}{\mu},
\]
\[
\mathrm{Var}[S(t)] = 2\frac{\lambda t}{\mu^2}.
\]

\section{Parameter Sensitivity}

\subsection*{Effect of $\lambda$}
\begin{itemize}
    \item Higher $\lambda$: more frequent jumps.
    \item $S(t)$ increases faster.
    \item Higher variance.
\end{itemize}

\subsection*{Effect of $\mu$}
\begin{itemize}
    \item Higher $\mu$: smaller mean jump size.
    \item Smaller $\mu$: larger jump sizes and heavier tail.
\end{itemize}

These behaviors are visualized interactively through the R Shiny app.

\section{Full R Shiny Code}

\subsection*{R Script}

\begin{lstlisting}[language=R]
# -------------------------------------------------------------
# Load Packages
# -------------------------------------------------------------
library(shiny)
library(ggplot2)

# -------------------------------------------------------------
# UI
# -------------------------------------------------------------
ui <- fluidPage(
  titlePanel("Compound Poisson Process Simulator"),

  sidebarLayout(
    sidebarPanel(
      numericInput("lambda", "Arrival Rate (λ)", value = 1, min = 0.1),
      numericInput("mu", "Exponential Rate (µ)", value = 1, min = 0.1),
      numericInput("tmax", "Simulation Horizon (t)", value = 10, min = 1),
      actionButton("go", "Run Simulation")
    ),

    mainPanel(
      plotOutput("histPlot"),
      plotOutput("pathPlot")
    )
  )
)

# -------------------------------------------------------------
# Server
# -------------------------------------------------------------
server <- function(input, output) {

  observeEvent(input$go, {

    # Simulate N(t)
    N_t <- rpois(1, input$lambda * input$tmax)

    # Simulate X_i
    X <- rexp(N_t, rate = input$mu)

    # Compute S(t)
    S_t <- cumsum(X)

    # Event times
    event_times <- sort(runif(N_t, 0, input$tmax))

    # Histogram of jumps
    output$histPlot <- renderPlot({
      df <- data.frame(Jumps = X)
      ggplot(df, aes(x = Jumps)) +
        geom_histogram(bins = 30) +
        ggtitle("Histogram of Jump Sizes X_i")
    })

    # S(t) step plot
    output$pathPlot <- renderPlot({
      df <- data.frame(time = event_times, S = S_t)
      ggplot(df, aes(x = time, y = S)) +
        geom_step() +
        ggtitle("Compound Poisson Process Path: S(t)")
    })
  })
}

# -------------------------------------------------------------
# Run the App
# -------------------------------------------------------------
shinyApp(ui = ui, server = server)
\end{lstlisting}


\begin{figure}[h!]
\centering
\includegraphics[width=0.7\textwidth]{hist1.png}
\caption{Histogram of Exponential Jump Sizes $X_i$}
\label{fig:hist_jumps}
\end{figure}

\begin{figure}[h!]
\centering
\includegraphics[width=0.7\textwidth]{download.png}
\caption{Step Path of Compound Poisson Process $S(t)$}
\label{fig:compound_path}
\end{figure}

\section{Conclusion}
This document described the compound Poisson process with exponential jump sizes, derived its distribution, analyzed parameter sensitivity, and provided a complete R Shiny implementation to simulate and visualize the process.

\end{document}
